% !TEX root = ../thesis-example.tex
%
\chapter{Schluss}
\label{sec:schluss}

Die erstellte Erweiterung enthält alle notwendigen Operatoren um syntaktische NLP-Parser zu evaluieren und zu vergleichen. Die Parser-Operatoren können auch zum Zweck der Annotation eingesetzt werden, da ihre Ausgabe mit Hilfe der Text-Processing-Erweiterung als Datei abgespeichert werden kann. Für das Einbringen von weiteren Parsern ist, durch die bestehenden Operatoren eine Vorlage gegeben.

Des weiteren bietet dieses Rahmenwerk einige Punkte zur Weiterentwicklung.
\begin{description}
\item[Zeitmessung]
Für die Bewertung der Parser kann neben der Korrektheit auch noch die Geschwindigkeit einbezogen werden. 
\item[Modelle trainieren]
Oft bieten die Parser die Möglichkeit ein neues Modell anhand einer Treebank zu trainieren. Die Umsetzung dieses Features mittels RapidMiner-Operatoren würde das Evaluations-Rahmenwerk enorm bereichern. 
\item[Vorverarbeitung der Ergebnisse]
Für den Vergleich von Parser-Ausgabe und Goldstandard werden beispielsweise in \cite{parseval} Möglichkeiten zur Bearbeitung der annotierten Sätze vorgestellt. Diese können ins Rahmenwerk eingebracht, und dann mittels Parameter aktiviert oder deaktiviert werden.
\item[Dependenz-Parser]
Das Rahmenwerk beinhaltet aktuell nur Parser, welche die Konstituenten der Sätze errechnen. Sogenannte Dependenz-Parser geben die Abhängigkeiten der Wörter untereinander an. Eine mögliche Erweiterung ist das Einbringen des Vergleichs von Dependenz-Parsern.
\end{description}