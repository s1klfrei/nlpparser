% !TEX root = ../thesis-example.tex
%
\pdfbookmark[0]{Abstract}{Abstract}
\chapter*{Abstract (Deutsch)}
\label{sec:abstract}
\vspace*{-10mm}

Der Teilbereich \textit{Natural Language Processing} der Informatik versucht dem Computer das sprachliche Wissen über die menschliche Sprache zu vermitteln. Speziell die Aufgabe des Syntax-Parsings, also das Zuweisen von Syntaktischen Informationen zu natürlich-sprachlichen Texten wird in dieser Arbeit betrachtet. \\
Ziel ist die Entwicklung eines Evaluations-Rahmenwerks für Syntax-Parser. Als Plattform wurde RapidMiner gewählt.\\
In dieser Arbeit werden Grundlagen der Computerlinguistik und Techniken zum Parsen vorgestellt. Weiter wird sowohl das Konzept, als auch die Implementierung des Rahmenwerks aufgeführt. Mit Hilfe der entwickelten RapidMiner-Operatoren werden der Berkeley-, Stanford- und OpenNLP-Parser evaluiert.\\
Mit dem resultierenden Rahmenwerk kann für einen Parser und einen Goldstandard Bewertungsmetriken ausgerechnet werden. Außerdem können die Bewertungen mehrerer Parser verglichen werden.
\vspace*{20mm}

{\usekomafont{chapter}Abstract (English)}\label{sec:abstract-diff} \\

The computer science subdivision called Natural Language Processing tries to impart the knowledge of the human language to the computer. The task of assigning syntactic information to texts in natural language, also called syntax parsing, is considered in this work.\\
The goal is to develop an evaluation framework for syntax parser. The framework will be written for RapidMiner.\\
This thesis presents basics of the Natural Language Processing and some methods for syntax parsing. Furthermore, both the concept and the implementation of the framework are specified. Using the developed RapidMiner operators, the Berkely, Stanford and OpenNLP parser are evaluated.\\
With the resulting framework it is possible to calculate valuation metrics for a parser and a goldstandard. The comparison of multiple parser ratings is possible.