% !TEX root = ../thesis-example.tex
%
\chapter{Einleitung}
\label{sec:intro}

Die Computerlinguistik, im Englischen \textit{Natural Language Processing} (kurz \textit{NLP}), ist das Teilgebiet der Informatik, welches sich damit beschäftigt, die Sprachbarriere zwischen Mensch und Maschine zu überwinden. Hierfür werden die Methoden und das Wissen von Sprachwissenschaft und Informatik zusammengeführt. \\
Bei der maschinellen Verarbeitung eines natürlich-sprachlichen Textes spielt die Syntax-Analyse eine große Rolle. Für diese Analyse werden durch sogenannte Syntax-Parser syntaktische Informationen zum eigentlichen Text hinzugefügt. Dieses syntaktische Wissen kann zum einen direkt von Anwendungen genutzt werden, wie etwa zur Prüfung von Sätzen auf grammatikalische Korrektheit. Zum anderen werden im Rahmen der semantischen Analyse eines Textes diese syntaktischen Informationen vorausgesetzt.\\
Natürlich-sprachliche Sätze sind oft mehrdeutig, was heißt, dass ihnen mehrere syntaktische Strukturen zugeordnet werden können. Damit können Parser für einen Satz verschiedene Ergebnisse errechnen. Daraus ergibt sich von selbst der Bedarf nach einer Evaluierung dieser Parser-Ausgaben. Eine Aussage über die Leistung der Parser ist besonders für die Weiterverarbeitung der Ergebnisse relevant. \cite{cl}

\section{Problemstellung der Arbeit}

Ziel dieser Arbeit ist das Implementieren eines Rahmenwerks zur Evaluierung von Syntax-Parsern einer natürlichen Sprache. Als natürliche Sprache wurde für diese Arbeit Englisch gewählt. Das Rahmenwerk soll für die Plattform RapidMiner \cite{rmstudio} geschrieben werden. Zur Umsetzung dieses Rahmenwerks soll eine Erweiterung von RapidMiner entwickelt werden, welche sowohl Parser zur Verfügung stellt, als auch die Möglichkeit zur Bewertung der Leistung der Parser anbietet. 

Diese Anforderung werden mittels RapidMiner-Operatoren umgesetzt. Somit bietet die resultierende Erweiterung einen Evaluationsoperator, drei Parser-Operatoren und einen Operator zum Vergleich der einzelnen Parser-Ergebnisse.

\section{Aufbau der Arbeit}
\begin{description}
\item[Kapitel \ref{sec:nlp}] stellt die Grundlagen des \textit{Natural Language Processings} und Techniken zum Parsen vor.
\item[Kapitel \ref{sec:konzept}] zeigt den konzeptuellen Aufbau des Rahmenwerks.
\item[Kapitel \ref{sec:impl}] beschreibt, wie die einzelnen Operatoren der RapidMiner-Erweiterung implementiert wurden.
\item[Kapitel \ref{sec:eval}] evaluiert die Leistung der eingebrachten Parser.
\item[Kapitel \ref{sec:related}] stellt ein NLP-Rahmenwerk einer anderen Plattform vor.
\item[Kapitel \ref{sec:schluss}] gibt zum Abschluss ein Fazit und einen Ausblick.
\end{description}