% !TEX root = ../thesis-example.tex
%
\chapter{Einleitung}
\label{sec:intro}


\section{Problemstellung der Arbeit}

Ziel dieser Arbeit ist das Implementieren eines Rahmenwerks zum Evaluierung von Syntax-Parsern einer natürlichen Sprache. Als natürliche Sprache wurde für diese Arbeit Englisch gewählt. Das Rahmenwerk soll für die Plattform RapidMiner \cite{rmstudio} geschrieben werden. Zur Umsetzung dieses Rahmenwerks soll eine Erweiterung von RapidMiner entwickelt werden, welche sowohl Parser zur Verfügung stellt, als auch die Möglichkeit zur Bewertung deren Leistung anbietet. 

Diese Anforderung werden mittels RapidMiner-Operatoren umgesetzt. Somit bietet die resultierende Erweiterung einen Evaluationsoperator, drei Parser-Operatoren und einen Operator zum Vergleich der einzelnen Parser-Ergebnisse.

\section{Aufbau der Arbeit}
\begin{description}
\item[Kapitel \ref{sec:nlp}] stellt die Grundlagen des \textit{Natural Language Processings} und Techniken zum Parsen vor.
\item[Kapitel \ref{sec:konzept}] zeigt den konzeptuellen Aufbau des Rahmenwerks.
\item[Kapitel \ref{sec:impl}] beschreibt wie die einzelnen Operatoren der RapidMiner-Erweiterung implementiert wurden.
\item[Kapitel \ref{sec:eval}] evaluiert die Leistung der eingebrachten Parser.
\item[Kapitel \ref{sec:related}] stellt ein NLP-Rahmenwerk einer anderen Plattform vor.
\end{description}